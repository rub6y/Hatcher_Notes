% In this file you should put the actual content of the blueprint.
% It will be used both by the web and the print version.
% It should *not* include the \begin{document}
%
% If you want to split the blueprint content into several files then
% the current file can be a simple sequence of \input. Otherwise It
% can start with a \section or \chapter for instance.

\chapter{Fundamental Group of Circle}

    \section{Homotopy Definitions}

    In this section we provide all the definition , lemmas and theorems regarding homotopies. 
    At the end we provide the definition of Fundamental Group of Topological Space and proof that it
    has a group structure.

    \begin{definition}[Homotopy of maps]
        \label{def:homotopy}
        
        Let $X,Y$ be topological spaces. We say that maps $f,g : X \to Y$ are homotopic ($f \simeq g$) iff there exists a continoues map $H : X \times I \to Y$ such that
        for any $x \in X$ 
            $$H(x,0) =  f(x) \text{ and } H(x,1) = g(x)$$.
    \end{definition}

    \begin{definition}[Path]
        \label{def:path}

        A path between points $x,y \in X$ is a continoues function $\gamma : I \to X$ such that 
            $$\gamma(0) = x \text{ and } \gamma(1) = y$$

    \end{definition}

    \begin{definition}[Loop]
        \label{def:loop}
        \uses{def:path}

        A loop is a path where $x = y$.
    \end{definition}

    \begin{definition}[Homotopy of Paths]
        \label{def:path_homotopy}
        \uses{def:path, def:homotopy}

        We say that two paths $\gamma_1, \gamma_2$ ($\gamma_1 \simeq_p \gamma_2$) from $x$ to $y$ are homotopic iff there exists homotopy map $H : I \times I \to X$
        such that H is homotopy of $\gamma_1, \gamma_2$ and for all $t \in I$ function $H(\cdot, t)$ is a path from $x$ to $y$.
    \end{definition}

    \begin{lemma}[All paths from $x$ to $y$ in $\mathbb{R}^n$ are Homotopic]
        \label{lem:Rn_path_equiv_lemma}
        \uses{def:path_homotopy}

        Any two paths $\gamma_1, \gamma_2$ from $x$ to $y$ in $\mathbb{R}^n$ are homotopic.
    \end{lemma}

    \begin{theorem}[Homotopy is equvialence relation]
        \label{thm:homotopy_equiv}
        \uses{def:homotopy}

        Relation $\simeq$ is an equvialence relation.
    \end{theorem}

    \begin{theorem}[Homotopy of paths is equvialence relation]
        \label{thm:path_homotopy_equiv}
        \uses{thm:homotopy_equiv, def:path_homotopy}

        Relation $\simeq_p$ of paths is an equvialence relation.
    \end{theorem}

    \begin{definition}[Composition of paths]
        \label{def:path_composition}
        \uses{def:path}

        Given to paths $\gamma_1, \gamma_2$ we definte $\gamma_1 \cdot \gamma_2$ by the formula:

        $$
            \gamma_1 \cdot \gamma_2 (t) =
            \begin{cases}
            \gamma_1(2s), & \text{if } t \leq \frac{1}{2}, \\
            \gamma_2(1-2s), & \text{if } t \geq \frac{1}{2}
            \end{cases}
        $$
    \end{definition}

    \begin{lemma}[Composition of paths is a path]
        \label{lem:path_comp_path}
        \uses{def:path,def:path_composition}

        Composition of paths is a path (The map given by \ref{def:path_composition} is continoues) 
    \end{lemma}

    \begin{lemma}[Composition of paths depend on homotopy class]
        \label{lem:path_comp_homoclass}
        \uses{def:path, def:path_composition, def:path_homotopy,thm:path_homotopy_equiv}

        If $f_0 \simeq_p f_1$ and $g_0 \simeq_p g_1$ then $f_0 \cdot g_0 \simeq_path f_1 \cdot g_1$ 
    \end{lemma}

    \begin{theorem}[Homotopy of loops is equvialence relation]
        \label{thm:loop_homotopy_equiv}
        \uses{thm:path_homotopy_equiv, def:path_homotopy}

        Relation $\simeq_l$ of loops is an equvialence relation. (We use $\simeq$ to simplify notation)
    \end{theorem}

    \begin{lemma}[Composition of loops is a loop]
        \label{lem:loop_comp_loop}
        \uses{def:loop, lem:path_comp_path}
        
        Composition of loops is a loop.
    \end{lemma}

    \begin{lemma}[Composition of loops depend on homotopy class]
        \label{lem:loop_comp_homoclass}
        \uses{lem:loop_comp_loop, lem:path_comp_homoclass, thm:loop_homotopy_equiv}

        If $f_0 \simeq_p f_1$ and $g_0 \simeq_p g_1$ then $f_0 \cdot g_0 \simeq_p f_1 \cdot g_1$ 
    \end{lemma}

    \begin{definition}[Fundamental Group]
        \label{def:fundamental_group}
        \uses{lem:loop_comp_homoclass}

        We definie the fundamental group of $(\pi_1(X, x_0),\cdot)$ as the set of equvialence classes of relation $\simeq$ with 
        the operation $\cdot$ - composition of loops
    \end{definition}

    \begin{lemma}[Composition is associative]
        \label{lem:loop_comp_assoc}
        \uses{def:fundamental_group}

        The operation $\cdot$ is associative.
    \end{lemma}

    \begin{lemma}[Composition has natural element]
        \label{lem:loop_comp_neutral}
        \uses{def:fundamental_group}
        
        There is an neutral element of $\cdot$, which is $[\text{const}_{x_0}]_{\simeq}$
    \end{lemma}

    \begin{lemma}[Composition has inverse]
        \label{lem:loop_comp_inv}
        \uses{def:fundamental_group}

        For every element of $\pi_1(X, x_0)$ there exists an inverse such that: 

        $[f] \cdot [g] = [\text{const}_{x_0}]$
    \end{lemma}
    
    \begin{theorem} [Fundamental Group is a Group]
        \label{thm:fundamental_group_is_group}
        \uses{lem:loop_comp_inv, lem:loop_comp_neutral, lem:loop_comp_assoc}

        The fundamental group is a group
    \end{theorem}

    \begin{theorem} [Fundamental Group of $\mathbb{R}^n$]
        \label{thm:Rn_fundamental_group}
        \uses{thm:fundamental_group_is_group, lem:Rn_path_equiv_lemma}
        
        The fundamental group of $\mathbb{R}^n$ is trivial
    \end{theorem}